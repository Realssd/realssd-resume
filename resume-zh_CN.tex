% !TEX TS-program = xelatex
% !TEX encoding = UTF-8 Unicode
% !Mode:: "TeX:UTF-8"

\documentclass{resume}
\usepackage{zh_CN-Adobefonts_external} % Simplified Chinese Support using external fonts (./fonts/zh_CN-Adobe/)
% \usepackage{NotoSansSC_external}
% \usepackage{NotoSerifCJKsc_external}
% \usepackage{zh_CN-Adobefonts_internal} % Simplified Chinese Support using system fonts
\usepackage{linespacing_fix} % disable extra space before next section
\usepackage{cite}

\begin{document}
\pagenumbering{gobble} % suppress displaying page number

\name{鲁文澔}

\basicInfo{
  \email{luwenhao\_2001@outlook.com} \textperiodcentered\ 
  \phone{(+86) 13984573576} \textperiodcentered\ 
} 
\section{\faGraduationCap\  教育背景}
\datedsubsection{\textbf{北京航空航天大学}, 北京}{2019 -- 至今}
\textit{本科在读}\ 计算机科学与技术, GPA: 3.73/4.0, 预计2024年7月毕业


\section{\faUsers\ 实习/项目经历}
\datedsubsection{\textbf{中国建设银行贵州省分行} (贵州贵阳)}{2022年7月 -- 2022年9月}
\role{实习}{金融科技实习生}
\begin{itemize}
  \item 协助进行需求梳理
  \item 通过Sketch以及React框架进行原型设计
\end{itemize}

\datedsubsection{\textbf{北京字节跳动有限公司} (北京)}{2022年11月 -- 2023年4月}
\role{实习}{服务端研发实习生}
\begin{itemize}
  \item 音视频检测算法的工程化,改进计算密集型算法工程化方式,提高计算稳定性
  \item 音视频巡检服务,日均将80000条视频分三个维度计算后聚合落入Clickhouse表
  \item 图片素材库建设,建设供检测算法使用的MongoDB数据库与配套查询服务
\end{itemize}

\datedsubsection{\textbf{基于容器技术的半实物卫星网络仿真系统}}{2022年12月 -- 2023年4月}
\role{后端服务开发}{实验室项目}
\begin{onehalfspacing}
使用Docker容器模拟卫星路由器运行,通过半实物方式对大规模卫星网络进行仿真
\begin{itemize}
  \item 使用Docker Swarm实现分布式部署, 并编写Go语言服务进行数据聚合
  \item 使用go-tc进行流量控制实现网络状态的模拟
  \item 通过Docker API自动化配置节点
\end{itemize}
\end{onehalfspacing}

\datedsubsection{\textbf{Funtion as a Service平台}}{2023 年4月 -- 至今}
\role{后端开发, 架构设计}{软件工程团队项目}
\begin{onehalfspacing}
基于Kubernetes的Function as a Service平台
\begin{itemize}
  \item 基于自制镜像与用户代码自定义打包镜像并通过API部署至Kubernetes
  \item 利用go语言实现函数网关,进行请求快速转发
  \item 支持分布式服务的日志采集与聚合
\end{itemize}
\end{onehalfspacing}

% Reference Test
%\datedsubsection{\textbf{Paper Title\cite{zaharia2012resilient}}}{May. 2015}
%An xxx optimized for xxx\cite{verma2015large}
%\begin{itemize}
%  \item main contribution
%\end{itemize}

\section{\faCogs\ IT 技能}
% increase linespacing [parsep=0.5ex]
\begin{itemize}[parsep=0.5ex]
  \item 编程语言: Golang == C > Python > C++
  \item 开发组件与框架: MySQL, Redis, RocketMQ, Gin, React, Hertz, GORM/GEN
  \item 开发工具: 熟悉Git, Docker以及Linux系统的使用,  了解Kubernetes
\end{itemize}

\section{\faHeartO\ 获奖情况}
\datedline{北航校级优秀生(\%10)}{2020-2021学年}
\datedline{学业优秀二等奖学金}{2020-2021学年}
\datedline{美国大学生数学建模竞赛-Honorable Mention}{2021年3月}
\datedline{挑战杯”首都大学生创业计划竞赛“青力冬奥”专项赛金奖}{2022年5月}

%% Reference
%\newpage
%\bibliographystyle{IEEETran}
%\bibliography{mycite}
\end{document}
